% this is Advances in Astronomy and Space Physics tex-sample
% the following preamble cannot be changed accept extension of the packages list; 
% new commands for your own abbreviations can also be added here 
\documentclass[a4paper]{article}
\usepackage{epsfig,amsmath,amsfonts,amssymb,setspace,multirow,textcomp}
\usepackage[T1]{fontenc}
\textheight 23cm \textwidth 18cm \hoffset= 0mm \voffset= 0cm
\topmargin -1cm \oddsidemargin -8mm \evensidemargin 0mm \columnsep = 4ex
\pagestyle{myheadings}
\renewcommand{\abstractname}{}
\renewcommand{\refname}{\sc references}
\renewcommand{\figurename}{Fig.}
\makeatletter
\renewcommand{\@oddhead}{\textit{Advances in Astronomy and Space Physics} \hfil}
\renewcommand{\@evenfoot}{\hfil \thepage \hfil}
\renewcommand{\@oddfoot}{\hfil \thepage \hfil}
\makeatother
\let\oldthebibliography=\thebibliography
\let\endoldthebibliography=\endthebibliography
\renewenvironment{thebibliography}[1]{\begin{oldthebibliography}{#1}\setlength{\parskip}{0ex}\setlength{\itemsep}{0ex}}{\end{oldthebibliography}}
% end of preamble

\begin{document}
\fontsize{11}{11}\selectfont % the font size cannot be changed in any case!
%  insert your title, authors information and text instead of the one provided below
\title{Calculations of the number of stars on celestial sphere}
\author{\textsl{A.\,B.~Smith$^{1}$, C.\,D.~Ivanov$^{2}$}}
\date{\vspace*{-6ex}}
\maketitle
\begin{center} {\small $^{1}$Institute for Astronomy, University of Hawaii, 2680 Woodlawn Drive, Honolulu, HI 96822, USA\\
$^{2}$Taras Shevchenko National University of Kyiv, Glushkova ave., 4, 03127, Kyiv, Ukraine\\
{\tt smith.ab@ifa.hawaii.edu}}
\end{center}

\begin{abstract}
The abstract of your paper should be here.\\[1ex]
{\bf Key words:} stars, Galaxy, observations
\end{abstract}

\section*{\sc introduction}
\indent \indent From time immemorial ancient people looked at the sky and wanted to know how much stars are there in the sky. Thus our work, the results of which are presented below, is related to more than topical problem (see e.g. \cite{bean76}).

Actually, here there should be your own introduction.

\section*{\sc the method of calculations}

\indent \indent For our calculations we used the following formula 
\begin{equation}\label{form1}
N_{tot}=N_{S}+N_{N},
\end{equation}
where $N_{tot}$, $N_{S}$ and $N_{N}$ are the total number of stars, the numbers of stars we can see and the number of stars we have never seen correspondingly.

As we could not calculate the stars we had never seen we decided to calculate only the stars visible for us \cite{onion07}. Thus our formula (\ref{form1}) gives us only the lower limit of all the stars, as was mentioned in \cite{carrot89}.

We also tried to identify some stars. Firstly, we used the ancient map (see Fig.~\ref{fig1}), but did not succeed. Hence we decided to use the modern one, shown in Fig.~\ref{fig2} and successfully identified several stars (see Table~\ref{tab1}).

% as far as AASP has two-columns in its final version, try to fit your figures into the size of one column (width not more than 0.5\linewidth !) Pay attention to the size of axis titles: the fontsize should be readable after such ``fitting''.
\begin{figure}[!h]
\centering
\begin{minipage}[t]{.45\linewidth}
\centering
\epsfig{file = image1.eps,width = .85\linewidth}
\caption{The ancient sky map.}\label{fig1}
\end{minipage}
\hfill
\begin{minipage}[t]{.45\linewidth}
\centering
\epsfig{file = image2.eps,width = .85\linewidth}
\caption{The modern sky map.}\label{fig2}
\end{minipage}
\end{figure}

\begin{table}
 \centering
 \caption{The characteristics of some identified stars.}\label{tab1}
 \vspace*{1ex}
 \begin{tabular}{ccc}
  \hline
  name & color & brightness \\
  \hline
  star 1 & red & very bright \\
  star 2 & green & bright \\
  star 3 & yellow & not so bright\\
 \hline 
 \end{tabular}
\end{table}

\section*{\sc results and conclusions}
\indent \indent As we have no time and possibility to calculate even the stars we can see, and one of us felt asleep just after the midnight, we can conclude, that the number of stars is very big.

\section*{\sc acknowledgement}
\indent \indent We would like to thank the nature for sending us down the cloudless sky and our friend, who made coffee for us.

% read the instructions for the 
\begin{thebibliography}{3}
{\small
\bibitem{bean76} Bean~A.\,B. \& Tomato~E.\,F. 1976, Nature, 1, 15
\bibitem{carrot89} Carrot~C.\,D., Aubergine~E.\,F., Potato~G.\,H. et al. 1989, ApJ, 2, 555
\bibitem{onion07} Onion~J.\,K. 2007, `Stars of the North hemisphere', Vegetable University Press, New York
}
\end{thebibliography}
\end{document}
